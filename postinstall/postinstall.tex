\documentclass[11pt,notumble]{leaflet}
% notumble - keep the second page right-side-up


\usepackage[none]{hyphenat} % to remove hyphenation, instead justifying
\usepackage{multicol}       % for multicolumn bullet lists
\usepackage{textcomp}       % for copyleft symbol
\usepackage{amssymb}        % for checkboxes


\begin{document}

\begin{center} \LARGE
    Navigating Linux
\end{center}

If you're reading this document, then it's likely you've had some form of Linux
installed on your laptop by members of the UMBC Linux Users' Group. You have
dived into the unfamiliar, and we hope that this pamphlet can guide you through
your first few weeks.

On your computer is a freshly installed copy of
\begin{multicols}{2}
\begin{itemize}
    \item[$\square$] Ubuntu
    \item[$\square$] Linux Mint
    \item[$\square$] Fedora
    \item[$\square$] Archlinux
\end{itemize}
\end{multicols}
with the desktop environment
\begin{multicols}{2}
\begin{itemize}
    \item[$\square$] GNOME
    \item[$\square$] XFCE
    \item[$\square$] LXDE
    \item[$\square$] KDE
    \item[$\square$] Cinnamon
    \item[$\square$] Awesomewm
\end{itemize}
\end{multicols}

\vfill

\begin{figure}[hb]
    \centering
    \includegraphics[width=0.60\textwidth]{tux-bw}
\end{figure}

\pagebreak

\section{Installing Software}

Installing software in Linux is not like in Windows or Mac; almost all forms of
Linux are based around freely-packaged software in huge collections called
\emph{repositories}. Almost all of the software you might ever need to install
is available here: browsers, word processors, etc. To access these repositories,
you will often use a program such as Synaptic or Software Center called a
\emph{package manager}.

Exactly how to access these varies by your Linux distribution, but they will
often either be on the desktop or in the application launcher. Once you open
your package manager, you can search for the name of a program you want to
install, such as ``firefox''. Package names are rarely capitalized or include
version numbers. Once found, you can instruct your package manager to install
it, and the program will be automatically downloaded, installed, and configured,
all without manual intervention or adware.

% XXX: The adware clause might be a bit iffy.
% XXX: What should I install?

\vfill

\begin{center} \small 
    % Discreetly include the year, so we know how old printed our copies of this
    % flier are.
    \textcopyleft{} Copyleft \the\year{} UMBC Linux Users' Group \\
    No rights reserved.
\end{center}

\end{document}
