\documentclass[11pt,notumble]{leaflet}
% notumble - keep the second page right-side-up


\usepackage[none]{hyphenat} % to remove hyphenation, instead justifying
\usepackage{multicol}       % for multicolumn bullet lists
\usepackage{textcomp}       % for copyleft symbol
\usepackage{amssymb}        % for checkboxes
\usepackage{nameref}        % for named references to sections
\usepackage{wrapfig}        % for small QR codes

\begin{document}

\begin{center} \LARGE
    Navigating Linux
\end{center}

If you're reading this document, then it's likely you've had some form of Linux
installed on your laptop by members of the UMBC Linux Users' Group. You have
dived into the unfamiliar, and we hope that this pamphlet can guide you through
your first few weeks.

\section{Your Install}
\label{section:your-install}

On your computer is a freshly installed copy of
\begin{multicols}{2}
\begin{itemize}
    \item[$\square$] Ubuntu
    \item[$\square$] Linux Mint
    \item[$\square$] Fedora
    \item[$\square$] Archlinux
\end{itemize}
\end{multicols}
with the desktop environment
\begin{multicols}{2}
\begin{itemize}
    \item[$\square$] GNOME
    \item[$\square$] XFCE
    \item[$\square$] LXDE
    \item[$\square$] KDE
    \item[$\square$] Cinnamon
    \item[$\square$] Awesomewm
\end{itemize}
\end{multicols}

\vfill

\begin{figure}[hb]
    \centering
    \includegraphics[width=0.60\textwidth]{tux-bw}
\end{figure}

\pagebreak

\section{Installing Software}
\label{section:installing-software}

Installing software in Linux is not like in Windows or Mac; almost all forms of
Linux are based around freely-packaged software in huge collections called
\emph{repositories}. Almost all of the software you might ever need to install
is available here: browsers, word processors, etc. To access these repositories,
you will often use a program such as Synaptic or Software Center called a
\emph{package manager}.

Exactly how to access these varies by your Linux distribution, but they will
often either be on the desktop or in the application launcher. Once you open
your package manager, you can search for the name of a program you want to
install, such as ``firefox''. Package names are rarely capitalized or include
version numbers. Once found, you can instruct your package manager to install
it, and the program will be automatically downloaded, installed, and configured,
all without manual intervention or adware.

% XXX: The adware clause might be a bit iffy.
% XXX: What should I install?

\section{Common Software Equivalents}
\label{section:common-software-equivalents}

Some of the software that many people regularly use are Microsoft products, such
as Microsoft Word, Internet Explorer, and Adobe Photoshop. Although students
often now use online equivalents, such as Google Drive and Sheets, there are
also open source free software equivalents. Here are some common ones; if you
need help finding another replacement, see \nameref{section:getting-help}.

% This table will resize to the width of a column automatically while
% maintaining aspect ratio.
\resizebox{\textwidth}{!}{%
\begin{tabular}{lcl}
    Internet Explorer &$\rightarrow$& Mozilla Firefox, Chromium \\
    Microsoft Office  &$\rightarrow$& LibreOffice \\
    Microsoft Outlook &$\rightarrow$& Mozilla Thunderbird \\
    Adobe Photoshop   &$\rightarrow$& GIMP, Inkscape \\
    Notepad           &$\rightarrow$& GEdit
\end{tabular}
}

% XXX: What are some other common ones?

\section{Using Linux for CMSC Classes}
\label{section:linux-for-cmsc}

\section{Getting Help}
\label{section:getting-help}

Linux is developed and kept alive by an open, helpful community, and a wealth of
freely shared information. That means that no matter your question, help is
always available, you just need to know how to find it.

Knowing how to search the Internet to find answers for your specific question is
an important skill, and can require practice. Sometimes, searching for the
nature of your question, such as ``How do I access my email'' is not enough; you
also need to include keywords about the context of the problem. ``How do I
access my email on Ubuntu with GNOME'' is a much better search. The name of your
distribution and desktop environment are usually good starting points, and you
can find them at the front of this leaflet under \nameref{section:your-install}.

Searching for solutions to problems on the Internet will often lead you to
mailing lists and forums; these can be many years old, and conducted as either
help threads, or discussion sessions between experienced developers. It is
important to be aware of how the types of threads you find may or may not be
relevant to your question.

\begin{wrapfigure}{r}{0.3\textwidth}
    \vspace{-3em}
    \begin{center}
        \includegraphics[width=0.3\textwidth]{site-qr}
    \end{center}
    \vspace{-3em}
\end{wrapfigure}

The UMBC Linux Users' Group provides resources for Linux help, also. Aside from
this printed leaflet, we maintain a mailing list, which accepts questions and
requests for help. You can post to this list any time, and receive help soon. In
addition, we meet weekly in ITE234 on Wednesdays at noon. These meetings often
don't have a specific agenda, and just serve as a place for Linux users to get
and provide help and feedback. We welcome new members, will happily provide help
with any issues you might come across. A link to our website, which includes
instructions on how to join our mailing list, is encoded here.


\vfill
\begin{center} \small 
    % Discreetly include the year, so we know how old printed our copies of this
    % flier are.
    \textcopyleft{} Copyleft \the\year{} UMBC Linux Users' Group \\
    No rights reserved.
\end{center}

\end{document}
