\documentclass[11pt,notumble]{leaflet}
% notumble - keep the second page right-side-up

\usepackage[margin=1cm]{geometry}
\usepackage[none]{hyphenat} % to remove hyphenation, instead justifying
\usepackage{multicol}       % for multicolumn bullet lists
\usepackage{textcomp}       % for copyleft symbol
\usepackage{amssymb}        % for checkboxes
\usepackage{nameref}        % for named references to sections
\usepackage{wrapfig}        % for small QR codes

\begin{document}

\begin{center} \LARGE
    Navigating Linux
\end{center}

If you're reading this document, then it's likely you've had some form of Linux
installed on your laptop by members of the UMBC Linux Users' Group. You have
dived into the unfamiliar, and we hope that this pamphlet can guide you through
your first few weeks.

\section{Your Install}
\label{section:your-install}

On your computer is a freshly installed copy of
\begin{multicols}{2}
\begin{itemize}
    \item[$\square$] Ubuntu
    \item[$\square$] Linux Mint
    \item[$\square$] Fedora
    \item[$\square$] Archlinux
\end{itemize}
\end{multicols}
with the desktop environment
\begin{multicols}{2}
\begin{itemize}
    \item[$\square$] GNOME
    \item[$\square$] XFCE
    \item[$\square$] LXDE
    \item[$\square$] KDE
    \item[$\square$] Cinnamon
    \item[$\square$] Awesomewm
\end{itemize}
\end{multicols}

\vfill

\begin{figure}[hb]
    \centering
    \includegraphics[width=0.60\textwidth]{tux-bw}
\end{figure}

\pagebreak

\section{Installing Software}
\label{section:installing-software}

Installing software in Linux is not like in Windows or Mac; almost all forms of
Linux are based around freely-packaged software in huge collections called
\emph{repositories}. Almost all of the software you might ever need to install
is available here: browsers, word processors, etc. To access these repositories,
you will often use a program such as Synaptic or Software Center called a
\emph{package manager}.

Exactly how to access these varies by your Linux distribution, but they will
often either be on the desktop or in the application launcher. Once you open
your package manager, you can search for the name of a program you want to
install, such as ``firefox''. Package names are rarely capitalized or include
version numbers. Once found, you can instruct your package manager to install
it, and the program will be automatically downloaded, installed, and configured,
all without manual intervention or adware.

% XXX: The adware clause might be a bit iffy.
% XXX: What should I install?

\section{Common Software Equivalents}
\label{section:common-software-equivalents}

Some of the software that many people regularly use products such as Microsoft
Word, Internet Explorer, and Adobe Photoshop.  Here are some common open source
equivalents; if you need help finding another replacement, see
\nameref{section:getting-help}.

% This table will resize to the width of a column automatically while
% maintaining aspect ratio.
\resizebox{\textwidth}{!}{%
\begin{tabular}{lcl} \small
    Internet Explorer    &$\rightarrow$& Mozilla Firefox, Chromium \\
    Microsoft Office     &$\rightarrow$& LibreOffice \\
    Microsoft Outlook    &$\rightarrow$& Mozilla Thunderbird \\
    Media Player         &$\rightarrow$& VLC \\
    iTunes               &$\rightarrow$& Banshee \\
    Adobe Photoshop      &$\rightarrow$& GIMP, Inkscape \\
    Notepad              &$\rightarrow$& GEdit
\end{tabular}
}

% XXX: What are some other common ones?

\section{Using Linux for CMSC Classes}
\label{section:linux-for-cmsc}

The Computer Science curriculum at UMBC uses languages such as Python, C, and
C++. Students are normally instructed to use tools such as PuTTY to remotely
connect to the GL system, and develop their software there. GL also has Linux at
its core, so if you choose to investigate parts of your operating system, you
may find it similar.

Connecting remotely to GL from Linux does not require an external program such
as PuTTY\@; the standard protocol used to connect between Linux and Linux-alike
computers is called SSH, and stands for Secure SHell. You can connect to GL by
opening a terminal or a shell on your own computer, usually in your application
menu with an icon such as \verb+>_+, access a command line interface similar to
GL's. Once opened, you can enter \verb+ssh user@gl.umbc.edu+ where \verb+user+
is your email username, and you will be prompted to enter your password. Once
complete, you can execute commands on GL\@ as you normally would via PuTTY\@.

If you are slightly more advanced, you may also want to be aware that you can
develop software locally on your computer. We can't provide a guide here, for
lack of space, but you may want to look into visiting our meetings and asking
for assistance.

% % XXX: had to cut this paragraph for space
% If you encounter issues, and would like to pursue this use of Linux, please read
% about our meetings and our mailing list under \nameref{section:getting-help}.

\section{Getting Help}
\label{section:getting-help}

Linux is developed and kept alive by an open, helpful community, and a wealth of
freely shared information. That means that no matter your question, help is
always available, you just need to know how to find it.

Knowing how to search the Internet to find answers for your specific question is
an important skill, and can require practice. Sometimes, searching for the
nature of your question, such as ``How do I access my email'' is not enough; you
also need to include keywords about the context of the problem. ``How do I
access my email on Ubuntu with GNOME'' is a much better search. The name of your
distribution and desktop environment are usually good starting points, and you
can find them at the front of this leaflet under \nameref{section:your-install}.

Searching for solutions to problems on the Internet will often lead you to
mailing lists and forums; these can be many years old, and conducted as either
help threads, or discussion sessions between experienced developers. It is
important to be aware of how the types of threads you find may or may not be
relevant to your question.

\begin{wrapfigure}{r}{0.3\textwidth}
    \vspace{-3em}
    \begin{center}
        \includegraphics[width=0.3\textwidth]{site-qr}
        \vspace{-2em}
        \texttt{lug.umbc.edu}
    \end{center}
\end{wrapfigure}

The UMBC Linux Users' Group provides resources for Linux help, also. Aside from
this printed leaflet, we maintain a mailing list, which accepts questions and
requests for help. You can post to this list any time, and receive help soon. In
addition, we meet weekly in ITE234 on Wednesdays at noon. These meetings often
don't have a specific agenda, and just serve as a place for Linux users to get
and provide help and feedback. We welcome new members, will happily provide help
with any issues you might come across. A link to our website, which includes
instructions on how to join our mailing list, is encoded here.

\section{What is Free Software?}
\label{section:free-software}

The Free Software Foundation describes free software as follows.

\begin{quote}
    “Free software” means software that respects users' freedom and community.
    Roughly, it means that the users have the freedom to run, copy, distribute,
    study, change and improve the software. Thus, “free software” is a matter of
    liberty, not price. To understand the concept, you should think of “free” as
    in “free speech,” not as in “free beer”.
\end{quote}

The community surrounding Linux is founded in part on these principles, and we
have worked together to provide good free software for all to use. At the UMBC
Linux Users' Group, we believe that by using Linux as a tool to grow the free
software community, we can also increase the quality of that software. As more
and more computer users and software developers use Linux and the suite of tools
usually associated with it, more bugs will be found and fixed, and those tools
will become better and better.

In the spirit of free software, we provide all of this information with good
intentions and free of charge. You can find the source of this document on our
GitHub page, linked from our website (see \nameref{section:getting-help}), and
suggest changes and improvements there. Welcome to our community. We hope you'll
like it.

\pagebreak
% Begin back page

\begin{center} \Large
    UMBC Linux Users' Group \\
    Post-Install Information
\end{center}

\vfill
\begin{center} \small 
    % Discreetly include the year, so we know how old printed our copies of this
    % flier are.
    \textcopyleft{} Copyleft \the\year{} UMBC Linux Users' Group \\
    No rights reserved. \\
    Git Version \input{git-describe}
\end{center}

\end{document}
